\chapter*{ВВЕДЕНИЕ}
\addcontentsline{toc}{chapter}{ВВЕДЕНИЕ}

Генерация ландшафта -- это способ создания ландшафта <<без участия человека>> \cite{researchITMO}. Полученные таким образом объекты могут применяться в различных областях. К примеру, сгенерированные ландшафты использовались в научно фантастических фильмах в неземных пейзажах \cite{openGL}. Также данному методу находится применение в игровой индустрии. С развитием технологий требования к игровым мирам растут, особенно к их размерам и детализации. Для их создания вручную требуется очень много времени и ресурсов. Генерация ландшафта может если не решить, то серьёзно упростить решение этой задачи. У данного метода есть ещё одно преимущество над ручным созданием – возможность реализации полного редактирования игрового мира. Также сгенерированные ландшафты могут применяться при моделировании водной и термической эрозии \cite{algorithms}.

Цель работы: разработать программу для генерации ландшафта. Пользователю должны быть доступны следующие возможности: задание, изменение и сохранение параметров генерации (алгоритм генерации, диапазон высот, дальность и шаг отрисовки), задание и изменение положения источника света (источник света находится на бесконечности, положение задаётся углами), управление положением камеры (перенос, поворот). Исследовать зависимость скорости генерации ландшафта от алгоритма генерации.

\begin{enumerate}[label={\arabic*)}]
	\item проанализировать и выбрать алгоритмы для создания изображения и генерации ландшафта;
	\item спроектировать программу для генерации и отрисовки ландшафта;
	\item выбрать средства реализации спроектированной программы;
	\item исследовать зависимость скорости генерации ландшафта от алгоритма генерации.
\end{enumerate}