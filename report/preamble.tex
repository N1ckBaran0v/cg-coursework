% пакеты
\usepackage{csquotes}
\usepackage[utf8]{inputenc}
\usepackage[T1]{fontenc}

% Библиография
\NewBibliographyString{accessmode}
\DeclareFieldFormat{url}{\bibstring{accessmode}\addcolon\space\url{#1}}
\DeclareFieldFormat{title}{#1}
\DefineBibliographyStrings{russian}{
accessmode={Режим доступа:},
urlseen={дата обращения:}
}
% скобки после пунктов
\setenumerate[1]{label={ \arabic*)}}
\usepackage{enumitem}

% математика в листингах
\lstset{
mathescape=true
}

% \FloatBarrier
\usepackage{placeins}

% $ в листингах
\newcommand{\dlr}{\mbox{\textdollar}}

% Тире в itemize
\renewcommand\labelitemi{---}

% \foreach
\usepackage{pgffor}

% Длинные таблицы
\usepackage{longtable}

% Картинки
\newcommand{\img}[3] {
\begin{figure}[t!]
    \center{\includegraphics[height=#1]{inc/img/#2}}
    \caption{#3}
    \label{img:#2}
\end{figure}
}
\newcommand{\imgw}[3] {
\begin{figure}[h!]
    \center{\includegraphics[width=#1]{inc/img/#2}}
    \caption{#3}
    \label{img:#2}
\end{figure}
}
\newcommand{\boximg}[3] {
\begin{figure}[h]
    \center{\fbox{\includegraphics[height=#1]{inc/img/#2}}}
    \caption{#3}
    \label{img:#2}
\end{figure}
}

% АА в сокращенном виде

\newcommand{\uchapter}[2] {
	\chapter*{#1}
	\addcontentsline{toc}{chapter}{#1}
}

\newcommand{\usection}[2] {
	\section*{#1}
	\addcontentsline{toc}{section}{#1}
}

\usepackage{pdfpages}
\usepackage{makecell}
\usepackage{mathtools}

\lstset{ %
	language=java,                 % выбор языка для подсветки (здесь это Java)
	basicstyle=\small\ttfamily, % размер и начертание шрифта для подсветки кода
	numbers=none,               % где поставить нумерацию строк (слева\справа)
	numberstyle=\small,           % размер шрифта для номеров строк
	stepnumber=1,                   % размер шага между двумя номерами строк
	%	numbersep=5pt,                % как далеко отстоят номера строк от подсвечиваемого кода
	showspaces=false,            % показывать или нет пробелы специальными отступами
	showstringspaces=false,      % показывать или нет пробелы в строках
	showtabs=false,             % показывать или нет табуляцию в строках
	frame=single,              % рисовать рамку вокруг кода
	tabsize=4,                 % размер табуляции по умолчанию равен 2 пробелам
	captionpos=t,              % позиция заголовка вверху [t] или внизу [b] 
	breaklines=true,           % автоматически переносить строки (да\нет)
	breakatwhitespace=false, % переносить строки только если есть пробел
	escapeinside={\#*}{*)},   % если нужно добавить комментарии в коде
	abovecaptionskip=-5pt                   % Показать название подгружаемого файла
}