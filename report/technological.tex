\chapter{Технологическая часть}

\section{Средства реализации}

Для реализации программы был выбран язык программирования Java, так как он содержит все необходимые средства для реализации выбранных в результате проектирования алгоритмов и структур данных. 

Для тестирования был выбран фреймворк JUnit5, так как он имеет достаточный функционал для написания модульных тестов.

\section{Примеры реализаций алгоритмов}

В листингах~\ref{lst:cut1}-\ref{lst:cut3} представлена реализация алгоритма преобразования координат вершин ландшафта и отсечения перед отрисовкой.

\begin{lstlisting}[label=lst:cut1,caption=Реализация алгоритма подготовки полигонов к отрисовке (часть~1)]
@Override
public void visit(@NotNull PolygonalModel polygonalModel) {
	for (var polygon : polygonalModel) {
		drawPolygon(polygon);
	}
}

private void drawPolygon(Polygon polygon) {
	for (var elem : polygon) {
		var dot = elem.drawDot;
		if (!dot.isUsed) {
			dot.set(center, elem.realDot);
			cameraMatrix.transformVector(dot);
			frustumMatrix.transformVector(dot);
			if (dot.w >= focus) {
				maybeVisible.add(dot);
				dot.to3D();
			} else {
				invalid.add(dot);
			}
			dot.isUsed = true;
		} else {
			if (dot.w >= focus) {
				maybeVisible.add(dot);
\end{lstlisting}
\begin{lstlisting}[label=lst:cut2,caption=Реализация алгоритма подготовки полигонов к отрисовке (часть~2)]
			} else {
				invalid.add(dot);
			}
		}
	}
	var len = maybeVisible.size();
	for (var incorrect : invalid) {
		for (var i = 0; i < len; ++i) {
			maybeVisible.add(len, findDot(incorrect, maybeVisible.get(i)));
		}
	}
	triangulate(maybeVisible, polygon.color);
	invalid.clear();
	maybeVisible.clear();
}

private DrawVector findDot(DrawVector a, DrawVector b) {
	var t = (b.w - focus) / (b.w - a.w);
	var result = new DrawVector();
	var x = b.x * b.w;
	result.x = x + (a.x - x) * t;
	var y = b.y * b.w;
	result.y = y + (a.y - y) * t;
	result.w = focus;
	result.to3D();
	result.isUsed = true;
	result.brightness = b.brightness + (a.brightness - b.brightness) * t;
	return result;
}

private void triangulate(List<DrawVector> polygon, Color color) {
	if (polygon.size() < 3) {
		return;
	}
	var first = polygon.getFirst();
	var second = polygon.get(1);
	for (var i = 2; i < polygon.size(); ++i) {
\end{lstlisting}
\begin{lstlisting}[label=lst:cut3,caption=Реализация алгоритма подготовки полигонов к отрисовке (часть~3)]
		var third = polygon.get(i);
		drawStrategy.draw(first, second, third, color);
		second = third;
	}
}
\end{lstlisting}

В листингах~\ref{lst:z-buffer1}-\ref{lst:z-buffer2} представлена реализация алгоритма, использующего Z-буффер, и закраски Гуро.

\begin{lstlisting}[label=lst:z-buffer1,caption=Реализация алгоритма отрисовки полигона (часть~1)]
@Override
public void draw(@NotNull DrawVector d1, @NotNull DrawVector d2, @NotNull DrawVector d3, @NotNull Color color) {
	var x0 = Math.max(xMin, (int) Math.floor(Math.min(d1.x, Math.min(d2.x, d3.x))));
	var x1 = Math.min(xMax, (int) Math.ceil(Math.max(d1.x, Math.max(d2.x, d3.x))));
	var y0 = Math.max(yMin, (int) Math.floor(Math.min(d1.y, Math.min(d2.y, d3.y))));
	var y1 = Math.min(yMax, (int) Math.ceil(Math.max(d1.y, Math.max(d2.y, d3.y))));
	var divider = (d1.x - d2.x) * (d2.y - d3.y) - (d2.x - d3.x) * (d1.y - d2.y);
	for (var x = x0; x <= x1; ++x) {
		for (var y = y0; y <= y1; ++y) {
			var k1 = ((x - d2.x) * (y - d3.y) - (x - d3.x) * (y - d2.y)) / divider;
			var k2 = ((x - d3.x) * (y - d1.y) - (x - d1.x) * (y - d3.y)) / divider;
			var k3 = ((x - d1.x) * (y - d2.y) - (x - d2.x) * (y - d1.y)) / divider;
			if (k1 >= 0 && k1 <= 1 && k2 >= 0 && k2 <= 1 && k3 >= 0 && k3 <= 1) {
				var z = d1.z * k1 + d2.z * k2 + d3.z * k3;
				if (buffer[x][y] > z) {
					buffer[x][y] = z;
					if (z > 0.9) {
						color.setAlpha((int) ((1 - z) * 2550));
					}
\end{lstlisting}
\clearpage
\begin{lstlisting}[label=lst:z-buffer2,caption=Реализация алгоритма отрисовки полигона (часть~2)]
					var brightness = d1.brightness * k1 + d2.brightness * k2 + d3.brightness * k3;
					color.setBrightness(brightness);
					image.setPixel(x, y, color.getRGBWithBrightness());
					color.setAlpha(255);
				}
			}
		}
	}
}
\end{lstlisting}

В листингах~\ref{lst:gen1}-\ref{lst:gen4} представлена реализация алгоритма генерации ландшафта с помощью кригинга.

\begin{lstlisting}[label=lst:gen1,caption=Реализация алгоритма генерации ландшафта (часть~1)]
@Override
public void run() {
	synchronized (landscape) {
		needStop = false;
		var map = generateHeights();
		var polygons = generatePolygons(map);
		if (!needStop) {
			landscape.setPolygons(polygons);
			landscape.setInputHeightsMap(heights);
			landscape.setSideSize(sideSize);
			landscape.setSquareSize(squareSize);
			landscape.setStep(step);
			landscape.setNeedRecalculate(true);
		}
	}
}

private Map2D<Integer, PolygonVector> generateHeights() {
	var map = new HashMap2D<Integer, PolygonVector>();
	for (var i = 0; i <= sideSize && !needStop; i += step) {
		for (var j = 0; j <= sideSize && !needStop; j += step) {
			if (i % squareSize + j % squareSize > 0) {
				map.put(i, j, new PolygonVector(i, interpolate(i, j), j));
			} else {
\end{lstlisting}
\begin{lstlisting}[label=lst:gen2,caption=Реализация алгоритма генерации ландшафта (часть~2)]
				map.put(i, j, new PolygonVector(i, fastHeights[i / squareSize][j / squareSize], j));
			}
		}
	}
	return map;
}

private List<Polygon> generatePolygons(@NotNull Map2D<Integer, PolygonVector> heights) {
	var polygons = new LinkedList<Polygon>();
	for (var i = 0; i < sideSize && !needStop; i += step) {
		var near = heights.get(i, 0);
		var dx = heights.get(i + step, 0);
		var dz = near;
		var far = dx;
		for (var j = 0; j < sideSize && !needStop; j += step) {
			far = heights.get(i + step, j + step);
			dz = heights.get(i, j + step);
			var normal1 = Vector3D.getNormal(near, dx, far);
			if (normal1.y < 0) {
				normal1.inverse();
			}
			var normal2 = Vector3D.getNormal(near, dz, far);
			if (normal2.y < 0) {
				normal2.inverse();
			}
			normal1.normalize();
			normal2.normalize();
			polygons.add(new Polygon(near, dx, far, normal1, new Color(0, 255, 0)));
			polygons.add(new Polygon(near, dz, far, normal2, new Color(0, 255, 0)));
			near = dz;
			dx = far;
		}
	}
	return polygons;
}
\end{lstlisting}
\clearpage
\begin{lstlisting}[label=lst:gen3,caption=Реализация алгоритма генерации ландшафта (часть~3)]
private double interpolate(int x, int y) {
	var result = 0.0;
	if (x % squareSize == 0 || y % squareSize == 0) {
		result = fastHeights[x / squareSize][y / squareSize];
	} else {
		var matrix = copy(crigingMatrix);
		var answer = new double[matrix.length];
		var xi = 0;
		var yi = 0;
		for (var k = 0; k < matrix.length; ++k) {
			answer[k] = covariation(xi, yi, x, y);
			if (yi == sideSize) {
				yi = 0;
				xi += squareSize;
			} else {
				yi += squareSize;
			}
		}
		Gauss.solve(matrix, answer);
		var i = 0;
		var j = 0;
		for (var k = 0; k < matrix.length; ++k) {
			result += answer[k] * fastHeights[i][j];
			++j;
			if (j == size) {
				j = 0;
				++i;
			}
		}
	}
	return result;
}

private double covariation(int x0, int y0, int x1, int y1) {
	var h2 = (x1 - x0) * (x1 - x0) + (y1 - y0) * (y1 - y0);
	return 1 - Math.exp(-h2/radius2);
}

private double[][] copy(double[][] matrix) {
	var result = new double[matrix.length][matrix.length];
\end{lstlisting}

\begin{lstlisting}[label=lst:gen4,caption=Реализация алгоритма генерации ландшафта (часть~4)]
		for (var i = 0; i < matrix.length; ++i) {
		System.arraycopy(matrix[i], 0, result[i], 0, matrix.length);
	}
	return result;
}
\end{lstlisting}

\section{Интерфейс программы}

На рисунке~\ref{img:interface} представлен пример графического интерфейса программы.

\FloatBarrier
\includeimage
{interface} % Имя файла без расширения (файл должен быть расположен в директории inc/img/)
{f} % Обтекание (без обтекания)
{h} % Положение рисунка (см. figure из пакета float)
{1\textwidth} % Ширина рисунка
{Пример работы программы} % Подпись рисунка
\FloatBarrier

\section{Тестирование}

Было реализовано модульное тестирование с помощью фреймворка JUnit5. В качестве меры, используемой при тестировании программы, было выбрано покрытие кода, подсчёт которого производился автоматически средствами среды разработки IntelliJ Idea. Процент покрытия кода составил 18\%.

Для создания теста использовалась аннотация $@Test$, а для проверки результата использовались статические методы класса $Assertions$. Пример модульного теста приведён в листинге~\ref{lst:test}.
Результат выполнения модульных тестов представлен в листинге~\ref{lst:results}.

\begin{lstlisting}[label=lst:test,caption=Пример модульного теста]
@Test
void negativeHeights2() {
	try {
		heights.getLast().removeLast();
		new Generator(landscape, heights, sideSize, squareSize, step);
		fail();
	} catch (IllegalArgumentException ignored) {
	}
}
\end{lstlisting}

\begin{lstlisting}[label=lst:results,caption=Результат выполнения модульных тестов]
> Task :test

com.github.N1ckBaran0v.program.math.GaussTest

Test solve() PASSED
Test negative() PASSED

com.github.N1ckBaran0v.program.scene.GeneratorTest

Test negativeSquareSize() PASSED
Test run() PASSED
Test negativeHeights1() PASSED
Test negativeHeights2() PASSED
Test negativeHeights3() PASSED
Test negativeStep() PASSED
Test negativeSideSize() PASSED

SUCCESS: Executed 9 tests in 1s

\end{lstlisting}

\usection{Выводы}

В данном разделе были выбраны средства реализации, разработана спроектированная программа, представлен графический интерфейс и проведено тестирование.