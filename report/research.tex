\chapter{Исследовательская часть}

\section{Технические характеристики}

Технические характеристики устройства, на котором выполнялось исследование:

\begin{enumerate}[label=\arabic*.]
	\item Операционная система -- Ubuntu 24.04.1 LTS.
	\item Процессор -- 1th Gen Intel® Core™ i7-1165G7.
	\item Оперативная память -- 16 Гб.
\end{enumerate}

\section{Зависимость скорости генерации ландшафта от параметров генерации}

Для замера процессорного времени использовался метод $getCurrentThreadCpuTime$ класса $ThreadMXBean$ из пакета $java.lang.management$. При замерах зависимости скорости генерации от одного из параметров для других параметров выставлялись значения по умолчанию. Для размера ландшафта значением по умолчанию считалось 5000, для шага задания точек -- 1000, для шага интерполяции -- 100. 

Размеры ландшафта при получении зависимости скорости генерации от размера были взяты в диапазоне от 1000 до 10000 с шагом 1000. Результаты приведены в таблице~\ref{tab:bench1}. График зависимости времени генерации ландшафта от размера приведён на рисунке~\ref{img:side_size}. Также на графике присутствует аппроксимация полинома 8 степени.

Для получения зависимости скорости генерации от шага задания точек были проведены замеры на значениях 100, 500, 1000, 2500 и 5000. Результаты приведены в таблице~\ref{tab:bench2}. График зависимости времени генерации ландшафта от шага задания точек приведён на рисунке~\ref{img:square_size}. Также на графике присутствует аппроксимация функции, обратной к полиному 4 степени.

Для получения зависимости скорости генерации от шага интерполяции были проведены замеры на значениях 50, 100, 250, 500 и 1000. Результаты приведены в таблице~\ref{tab:bench3}. График зависимости времени генерации ландшафта от шага задания точек приведён на рисунке~\ref{img:step}. Также на графике присутствует аппроксимация функции, обратной к полиному 4 степени.

\clearpage

\begin{longtable}{|p{.5\textwidth - 2\tabcolsep}|p{.5\textwidth - 2\tabcolsep}|}
	\caption{\label{tab:bench1}Зависимость времени генерации ландшафта от размера} \\
	\hline
	\makecell{Размер стороны ландшафта} & \makecell{Время генерации, c} \\  
	\hline
	\makecell{1000} & \makecell{0.000674202} \\  
	\hline
	\makecell{2000} & \makecell{0.002533752} \\  
	\hline
	\makecell{3000} & \makecell{0.006030571} \\  
	\hline
	\makecell{4000} & \makecell{0.022136019} \\  
	\hline
	\makecell{5000} & \makecell{0.065762955} \\  
	\hline
	\makecell{6000} & \makecell{0.226550813} \\  
	\hline
	\makecell{7000} & \makecell{0.572844355} \\  
	\hline
	\makecell{8000} & \makecell{1.583992556} \\  
	\hline
	\makecell{9000} & \makecell{3.050565931} \\  
	\hline
	\makecell{10000} & \makecell{7.041135699} \\  
	\hline
\end{longtable}

\FloatBarrier
\includeimage
{side_size} % Имя файла без расширения (файл должен быть расположен в директории inc/img/)
{f} % Обтекание (без обтекания)
{h} % Положение рисунка (см. figure из пакета float)
{1\textwidth} % Ширина рисунка
{Зависимость времени генерации ландшафта от размера} % Подпись рисунка
\FloatBarrier

\clearpage

\begin{longtable}{|p{.5\textwidth - 2\tabcolsep}|p{.5\textwidth - 2\tabcolsep}|}
	\caption{\label{tab:bench2}Зависимость времени генерации ландшафта от шага задания точек} \\
	\hline
	\makecell{Шаг задания точек} & \makecell{Время генерации, c} \\  
	\hline
	\makecell{100} & \makecell{0.309796580} \\  
	\hline
	\makecell{500} & \makecell{1.388772992} \\  
	\hline
	\makecell{1000} & \makecell{0.065925029} \\  
	\hline
	\makecell{2500} & \makecell{0.006472606} \\  
	\hline
	\makecell{5000} & \makecell{0.002810549} \\  
	\hline
\end{longtable}

\FloatBarrier
\includeimage
{square_size} % Имя файла без расширения (файл должен быть расположен в директории inc/img/)
{f} % Обтекание (без обтекания)
{h} % Положение рисунка (см. figure из пакета float)
{1\textwidth} % Ширина рисунка
{Зависимость времени генерации ландшафта от шага задания точек} % Подпись рисунка
\FloatBarrier

\clearpage

\begin{longtable}{|p{.5\textwidth - 2\tabcolsep}|p{.5\textwidth - 2\tabcolsep}|}
	\caption{\label{tab:bench3}Зависимость времени генерации ландшафта от шага интерполяции} \\
	\hline
	\makecell{Шаг задания точек} & \makecell{Время генерации, c} \\  
	\hline
	\makecell{50} & \makecell{0.290930077} \\  
	\hline
	\makecell{100} & \makecell{0.065483174} \\  
	\hline
	\makecell{250} & \makecell{0.007514206} \\  
	\hline
	\makecell{500} & \makecell{0.000917739} \\  
	\hline
	\makecell{1000} & \makecell{0.000075908} \\  
	\hline
\end{longtable}

\FloatBarrier
\includeimage
{step} % Имя файла без расширения (файл должен быть расположен в директории inc/img/)
{f} % Обтекание (без обтекания)
{h} % Положение рисунка (см. figure из пакета float)
{1\textwidth} % Ширина рисунка
{Зависимость времени генерации ландшафта от шага интерполяции} % Подпись рисунка
\FloatBarrier

\clearpage

\usection{Выводы}

В данном разделе было проведено исследование зависимости скорости генерации ландшафта от параметров генерации. При увеличении размера ландшафта время его генерации растёт как полином 8 степени, а при увеличении шага задания точек и шага интерполяции - убывает как функция, обратная к полиному 4 степени. Это не относится к случаю, когда шаг задания точек и шаг интерполяции совпадают, так как в данном случае интерполяция не проводится.